\section{Progressão Aritmética}

\begin{definition}
Uma \textdef{progressão aritmética} (ou simplesmente \textdef{PA}) é uma sequência na qual a diferença entre um termo e seu anterior (exceto
quando o termo em questão é o primeiro) é constante. Essa diferença constante é chamada de \textdef{razão} da progressão e representada pela
letra $r$.
\end{definition}

\begin{remark}
De maneira recursiva, o $n$-ésimo, $n>1$, termo de uma PA é escrito
como:
%
\begin{equation*}
a_n = a_{n-1} + r.
\end{equation*}
\end{remark}

\begin{example}
Uma fábrica de automóveis produziu 400 veículos em janeiro e aumenta mensalmente sua produção em 30 veículos. Quantos veículos foram produzidos em junho?
\end{example}

\begin{solution}
A produção de cada mês será, em número de veículos:
%
\begin{itemize}
	\item Janeiro -- 400;
	\item Fevereiro -- 430;
	\item Março -- 460;
	\item Abril -- 490;
	\item Maio -- 520;
	\item Junho -- 550.
\end{itemize}
%
Poderíamos ter obtido a produção de junho calculando $400 + 5 \ctimes 30$.
\end{solution}
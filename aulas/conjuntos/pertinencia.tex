\section{Pertinência}

Na seção anterior, entendemos o que é um conjunto e como podemos defini-lo. Agora, teremos o poder de relacionar objetos a conjuntos. A ideia é poder perguntar se um dado objeto faz parte de tal conjunto, o que pode, ou não, ser verdade.

\begin{definition}[Relação de Pertinência]
\label{def:in}
Dado um conjunto arbitrário $A$ e um objeto $x$,
poderemos sempre perguntar se $x$ é um elemento de $A$. Se for o caso, representaremos por $x \in A$; do contrário, por $x \notin A$.
\end{definition}

\begin{example}
Considere $PP$ e $V$ conforme definido nos Exemplos~\ref{ex-vogais} e~\ref{ex-primos-pares}, respectivamente. Temos que $\texttt{e} \in \mathcal{V}$ e $3 \notin PP$.
\end{example}

\begin{example}
Considere $L$ como sendo o conjunto de todos as letras do alfabeto latino.
Sabendo que as vogais do conjunto $V$ fazem parte desse alfabeto, podemos concluir que $\texttt{a}, \texttt{e}, \texttt{i}, \texttt{o}, \texttt{u} \in L$.
Também sabemos que a letra $\texttt{j}$ faz parte do alfabeto, isso é, $\texttt{j} \in L$.
Contudo, como $\texttt{j}$ não é uma vogal, podemos concluir que $\texttt{j} \notin V$.

\end{example}

\begin{example}
Conjuntos também podem pertencer a outros conjuntos.
Os habitantes de um país pode ser visto como um conjunto de pessoas.
Por sua vez, cada pessoa pode ser vista como um conjunto de células.
Mais tarde, na definição~\ref{def:powerset}, veremos um famoso conjunto formado por conjuntos.
\end{example}

\begin{definition}[O Conjunto Vazio]
\label{def:emptyset}
O conjunto que não possui elementos é chamado de \textdef{conjunto vazio} e é representado por $\emptyset$.
Em outras palavras, para qualquer que seja o objeto $x$, temos $x \notin \emptyset$.
\end{definition}

Agora, com a Definição~\ref{def:emptyset}, o Exercício~\homeworkref{exe:vazio-notacao} já pode ser feito.

\begin{example}
Quais outros conjuntos você conhece? Que tal pensar sobre o conjunto $A = \set{x \tq x \notin A}$?
\end{example}


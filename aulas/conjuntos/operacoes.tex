\section{Operações}
Assim como na aritmética, onde os números possuem suas operações (soma, subtração,  multiplicação, etc.), os conjutos também possuem suas operações.
De uma forma geral, operações têm o objetivo de receber objetos de um tipo, operá-lós, e resultar em algum outro objeto, podendo ser um diferente ou não.

Ainda aproveitando a aritmética, sabemos que o resultado da operação $1 + 5$ é o número $6$, ms não precisamos realizar a operação imediatamente, isto é, podemos utilizar o próprio $1 + 5$, a idéia é que $6$ (seis) é apenas um nome diferente e mais simples para $1 + 5$, visto que representam o mesmo número.

A mesma coisa acontece em conjutos, porém, nem sempre sabemos um nome mais simples para o resultado final da operação, e simplesmente deixamos do jeito que está.


\import{}{uniao_e_intersecao.tex}
\import{}{complementar.tex}
\import{}{diferenca.tex}


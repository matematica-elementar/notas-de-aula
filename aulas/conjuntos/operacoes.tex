\section{Operações}
Assim como na aritmética, onde os números possuem suas operações (soma, subtração,  multiplicação, etc.), os conjutos também possuem suas operações.
De uma forma geral, operações têm o objetivo de receber objetos de um tipo, operá-lós, e resultar em algum outro objeto, podendo ser um diferente ou não.

Ainda aproveitando a aritmética, sabemos que o resultado da operação $1 + 5$ é o número $6$, ms não precisamos realizar a operação imediatamente, isto é, podemos utilizar o próprio $1 + 5$, a idéia é que $6$ (seis) é apenas um nome diferente e mais simples para $1 + 5$, visto que representam o mesmo número.

A mesma coisa acontece em conjutos, porém, nem sempre sabemos um nome mais simples para o resultado final da operação, e simplesmente deixamos do jeito que está.

\subsection{Complementar}
A noção de complementar de um conjunto só faz sentido quando fixamos um \textdef{conjunto universo}, que denotaremos por $\U$. Uma vez fixado $\U$, todos os elementos considerados pertencerão a $\U$ e todos os conjuntos serão subconjuntos de $\U$. Na geometria plana, por exemplo, $\U$ é o plano.

\begin{definition}
Dado um conjunto $A$ (isto é, um subconjunto de $\U$), chama-se \textdef{complementar} de $A$ o conjunto $A^C$ formado pelos elementos de $\U$ que não pertencem a $A$.
\end{definition}

\begin{example}
Seja $\U$ o conjunto dos triângulos. Qual o complementar do conjunto dos triângulos escalenos?
\end{example}

\begin{proposition}[Propriedades do complementar]
\label{prop-complementar}
Fixado um conjunto universo $\U$, sejam $A$ e $B$ conjuntos. Tem-se:
%
\begin{enumerate}
	\item $\U^C=\emptyset$ e $\emptyset^C = \U$;
	\item $\prn{A^C}^C=A$ (todo conjunto é complementar do seu complementar);
	\item Se $A \subset B$, então $B^C \subset A^C$ (se um conjunto está contido em outro, seu complementar contém o complementar desse outro). 
\end{enumerate}
\end{proposition}

\begin{proof}
\begin{enumerate}
\item[]
\item A inclusão $\emptyset \subset \U^C$ é imediata pois $\emptyset$ é subconjunto de qualquer conjunto. 

Provemos, agora, que $\U^C \subset \emptyset$. Suponha, por absurdo, que $\U^C \notsubset \emptyset$; ou seja, existe $x \in \U^C$ tal que $x \notin \emptyset$. Ora, se $x \in \U^C$, então $x \notin \U$, o que é um absurdo. Logo, $\U^C \subset \emptyset$.

Dos fatos de que $\emptyset \subset \U$ e $\U \subset \emptyset$, conclui-se que $\U^C = \emptyset$.

A demonstração de que $\emptyset^C = \U$ fica como exercício para o leitor.

\item Provemos, primeiro, que $A \subset (A^C)^C$. Para tal, seja $x \in A$. Logo, $x \notin A^C$. Assim, $x \in (A^C)^C$. Então, $A \subset (A^C)^C$. A prova de que $(A^C)^C \subset A$, item restante para podermos concluir que a igualdade desejada é válida, fica como exercício para o leitor.

\item Sejam $A$, $B$ conjuntos tais que $A \subset B$. Além disso, seja $x \in B^C$, o que implica em $x \notin B$. Temos duas possibilidades para a presença de $x$ no conjunto $A$, a saber, $x \in A$ e $x \notin A$. Se $x \in A$, teríamos $x \in B$ também pois $A \subset B$; um absurdo visto que já temos $x \notin B$. Logo, concluímos que $x \notin A$, ou seja, $x \in A^C$. Segue, então, que $B^C \subset A^C$.
\end{enumerate}
\end{proof}

\begin{definition}
A \textdef{diferença} entre os conjuntos $A$ e $B$, denotada por $B\setminus A$, é definida por:
%
\[
B \setminus A = \set{x \tq x \in B \text{ e } x \notin A}
\]
\end{definition}

\begin{remark}
Em geral, não é verdade que $A \setminus B = B \setminus A$. Além disso, note que $A^C = \U\setminus A$.
\end{remark}


\subsection{Reunião e Interseção}

\begin{definition}
Dados os conjuntos $A$ e $B$, definem-se:
%
\begin{enumerate}
	\item A \textdef{reunião} $A \cup B$ como sendo o conjunto formado pelos elementos que pertencem a pelo menos um dos conjuntos $A$ e $B$;
	\item A \textdef{interseção} $A \cap B$ como sendo o conjunto formado pelos elementos que pertencem a ambos $A$ e $B$.
\end{enumerate}
\end{definition}

\begin{example}
Sejam $A = \set{1, 2, 3}$ e $ B = \set{2,5}$. Determine $A \cup B$, $A \cap B$, $A \setminus B$ e $B \setminus A$.
\end{example}

\begin{solution}
\begin{align*}
	&A \cup B = \{1,2,3,5\};\\
	&A \cap B = \{2\};\\
	&A \setminus B = \{1,3\};\\
	&B \setminus A = \{5\}.
\end{align*}
\end{solution}

\begin{proposition}[Propriedades da reunião e interseção]
Sejam $A$, $B$ e $C$ conjuntos. Tem-se:
\begin{enumerate}
  \item \textdef{Comutatividade}: $A \cup B = B \cup A$ e $A \cap B = B \cap A$;
  \item \textdef{Associatividade}: $\left(A \cup B \right) \cup C = A
  \cup \left( B \cup C \right)$ e $\left(A \cap B \right) \cap C = A
  \cap \left( B \cap C \right)$;
  \item \textdef{Distributividade}, de uma em relação à outra: $A \cap
  \left( B \cup C \right) = \left(A \cap B \right) \cup \left( A \cap C
  \right)$ e $A \cup \left( B \cap C \right) = \left(A \cup B \right) \cap
  \left( A \cup C  \right)$;
  \item $A \subset \prn{A \cup B}$ e $\prn{A \cap B} \subset A$;
  \item \textdef{Leis de DeMorgan}: $\left( A \cup B \right)^C = A^C \cap
  B^C$ e $\left(A \cap B \right)^C = A^C \cup B^C$.

  \end{enumerate}
\end{proposition}

% Proposição 21
\begin{proof}
\begin{enumerate}
	\item[]
	\item Exercício.
	\item Provemos que $A \cap (B \cap C) \subset (A \cap B) \cap C$. Para tal, seja $x \in A \cap (B \cap C)$, ou seja, $x \in A$ e $x \in B \cap C$. De $x \in B \cap C$, temos $x \in B$ e $x \in C$. Como $x \in A$ e $x \in B$, segue que $x \in A \cap B$. Além disso, $x \in C$. Então, $x \in A \cap (B \cap C)$. Logo, $A \cap (B \cap C) \subset (A \cap B) \cap C$. 

  A prova de que $A \cap (B \cap C) \supset (A \cap B) \cap C$, necessária para concluir a igualdade desejada, fica como exercício. Também o fica a verificação da comutatividade da reunião.
	\item Exercício.
	\item Exercício.
\end{enumerate}
\end{proof}

\begin{onlineact}[\khan{https://pt.khanacademy.org/math/statistics-probability/probability-library/basic-set-ops/e/basic_set_notation}{Notação Básica de Conjunto}]
Veja o desempenho na Missão O Mundo da Matemática - Probabilidade.
\end{onlineact}

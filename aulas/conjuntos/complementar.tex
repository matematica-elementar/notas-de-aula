\subsection{Complementar}
A noção de complementar de um conjunto só faz sentido quando fixamos um \textdef{conjunto universo}, que denotaremos por $\U$. Uma vez fixado $\U$, todos os elementos considerados pertencerão a $\U$ e todos os conjuntos serão subconjuntos de $\U$. Na geometria plana, por exemplo, $\U$ é o plano.

\begin{definition}[Complementar]
\label{def:complementar}
Dado um conjunto $A$ (isto é, um subconjunto de $\U$), chama-se \textdef{complementar} de $A$ o conjunto $A^C$ formado pelos elementos de $\U$ que não pertencem a $A$. Em outras palavras:
$$ x \in A^C \text{ se e somente se } x \notin A $$
\end{definition}

\begin{remark}
	Também é o caso que $ x \notin A^C \text{ se e somente se } x \in A $, como a forma contrapositiva da definição anterior.
\end{remark}

\begin{example}
Seja $\U$ o conjunto dos triângulos. Qual o complementar do conjunto dos triângulos escalenos?
\end{example}

\begin{proposition}[Propriedades do complementar]
\label{prop-complementar}
Fixado um conjunto universo $\U$, sejam $A$ e $B$ conjuntos. Tem-se:
%
\begin{enumerate}
	\item $\U^C=\emptyset$ e $\emptyset^C = \U$;
	\item $\prn{A^C}^C=A$ (todo conjunto é complementar do seu complementar);
	\item Se $A \subset B$, então $B^C \subset A^C$ (se um conjunto está contido em outro, seu complementar contém o complementar desse outro). 
\end{enumerate}
\end{proposition}

\begin{proof}
\begin{enumerate}
\item[]
\item A inclusão $\emptyset \subset \U^C$ é imediata pois $\emptyset$ é subconjunto de qualquer conjunto. 

Provemos, agora, que $\U^C \subset \emptyset$. Suponha, por absurdo, que $\U^C \notsubset \emptyset$; ou seja, existe $x \in \U^C$ tal que $x \notin \emptyset$. Ora, se $x \in \U^C$, então $x \notin \U$, o que é um absurdo. Logo, $\U^C \subset \emptyset$.

Dos fatos de que $\emptyset \subset \U$ e $\U \subset \emptyset$, conclui-se que $\U^C = \emptyset$.

A demonstração de que $\emptyset^C = \U$ fica como exercício para o leitor.

\item Provemos, primeiro, que $A \subset (A^C)^C$. Para tal, seja $x \in A$. Logo, $x \notin A^C$. Assim, $x \in (A^C)^C$. Então, $A \subset (A^C)^C$. A prova de que $(A^C)^C \subset A$, item restante para podermos concluir que a igualdade desejada é válida, fica como exercício para o leitor.

\item Sejam $A$, $B$ conjuntos tais que $A \subset B$. Além disso, seja $x \in B^C$, o que implica em $x \notin B$. Temos duas possibilidades para a presença de $x$ no conjunto $A$, a saber, $x \in A$ e $x \notin A$. Se $x \in A$, teríamos $x \in B$ também pois $A \subset B$; um absurdo visto que já temos $x \notin B$. Logo, concluímos que $x \notin A$, ou seja, $x \in A^C$. Segue, então, que $B^C \subset A^C$.
\end{enumerate}
\end{proof}



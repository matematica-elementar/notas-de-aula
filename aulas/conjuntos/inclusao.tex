\section{Inclusão}

\begin{definition}
Considere dois conjuntos $A$ e $B$.
Se for o caso que todo elemento de $A$ seja, também, elemento de $B$, diz-se que $A$ é um \textdef{subconjunto} de $B$, que $A$ está \textdef{contido} em $B$, ou que $A$ é \textdef{parte} de $B$.
Para indicar esse fato, usa-se a notação $A \subset B$.
\end{definition}

\begin{remark}
Quando $A$ não é um subconjunto de $B$, escreve-se $A \notsubset B$.
Na prática, isso significa que existe pelo menos um elemento que está em $A$ mas não está em $B$, em outras palavras, existe um elemento $a$ tal que $a \in A$ mas $a \notin B$.
\end{remark}

\begin{definition}
Quando $A \subset B$, dizemos que $B$ \textdef{contém} $A$, e escrevemos $B \supset A$.
\end{definition}

\begin{example}
Considere $T$ o conjunto de todos os triângulos e $P$ o conjunto dos polígonos no plano. Como todo triângulo é um polígono podemos concluir que $T \subset P$.
Observe também que $P \notsubset T$. Para poder concluir isso precisamos encontrar um elemento de $P$ que não seja um elemento de $T$. Ora, basta considerar um quadrado $q$ com lados de tamanho 1, como todo quadrado é um polígono, temos que $q \in P$, mas quadrados não são triangulos, então $q \notin T$.
\end{example}

\begin{example}
Na Geometria, uma reta, um plano e o espaço são conjuntos. Seus
elementos são pontos. Quando dizemos que uma reta $r$ está no plano $\Pi$, estamos afirmando que $r$ está contida em $\Pi$ ou, equivalentemente, que $r$ é um subconjunto de $\Pi$, pois todos os pontos que pertencem a $r$ pertencem, também, a $\Pi$. Nesse caso, deve-se escrever $ r \subset \Pi$. Porém, não é correto dizer que $r$ pertence a $\Pi$, nem escrever $r \in \Pi$. Os
elementos do conjunto $\Pi$ são pontos, não retas.
\end{example}

\begin{proposition}[Inclusão universal do $\emptyset$]
Seja um conjunto $A$, logo, podemos concluir que $\emptyset \subset A$. Em outras palavras, o conjunto vazio é subconjunto de todos os conjuntos.
\end{proposition}

\begin{proof}
Considere um conjunto $A$ arbitrário.
Suponha, para chegar num absurdo, que $\emptyset \notsubset A$.
Logo, podemos concluir que existe um elemento $x$ tal que $x \in \emptyset$ mas $x \notin A$.
O que nos leva a um absurdo, pois concluímos que há um elemento no $\emptyset$, mas pela sua definição, não existem elementos no $\emptyset$.
Portanto, podemos concluir que $\emptyset \subset A$.
\end{proof}

\begin{remark}
Ao manter a arbitrariedade de um conjunto, qualquer conclusão relacionada a este conjunto valerá para todos os conjuntos.
\end{remark}

\begin{definition}
Dizemos que $A$ é um \textdef{subconjunto próprio} de $B$ quando $A \subset B$ mas $A \neq B$. Quando isso ocorre utiliza-se a notação $A \subsetneq B$.
\end{definition}

\begin{proposition}[Propriedades da inclusão]
Considere $A$, $B$ e $C$ conjuntos arbitrários. Logo, são válidas as propriedades a seguir:
%
\begin{enumerate}
\item \textdef{Reflexividade}: $A \subset A$;
\item \textdef{Antissimetria}: Se $A \subset B$ e $B \subset A$, então $A = B$;
\item \textdef{Transitividade}: Se $A \subset B$ e $B \subset C$, então $A \subset C$.
\end{enumerate}
\end{proposition}

\begin{proof}
\begin{enumerate}

\item[] % Não enumerar o texto do \begin{proof}

\item
	Seja $x \in A$ um elemento arbitrário.
	Ora, como já temos que $x \in A$ podemos concluir que $A \subset A$.

\item
	Sejam $A$ e $B$ conjuntos tais que $A \subset B$ e $B \subset A$.
	Suponha, por contradição, que $A \ne B$, ou seja, existe $x \in A$ tal que $x \notin B$ (1) ou existe $x \in B$ tal que $x \notin A$ (2).
	Ora, (1) é o mesmo que $A \notsubset B$, contradizendo $A \subset B$.
	Analogamente, (2) contradiz $B \subset A$.
	Portanto, $A = B$.

\item
	Sejam $A$, $B$ e $C$ conjuntos tais que $A \subset B$ e $B \subset C$.
	Agora basta demonstrar que $A \subset C$.
	Para isso, considere $x \in A$ um elemento arbitrário.
	Como temos que $A \subset B$, podemos concluir que $x \in B$.
	E, como $x \in B$ e $B \subset C$, segue que $x \in C$.
	Portanto, $A \subset C$.

\end{enumerate}
\end{proof}

\begin{definition}
Dado um conjunto $A$, chamamos de \textdef{conjunto das partes} de $A$ o conjunto formado por todos os seus subconjuntos, e denotamo-lo $\mathcal{P}(A)$.
\end{definition}

\begin{example}
Dado $A = \set{1,2,3}$, determine $\powerset A$.
\end{example}

\begin{solution}
$\powerset A=\{\emptyset,\{1\}, \{2\}, \{3\}, \{1,2\}, \{2,3\}, \{1,3\},A\}$.
\end{solution}

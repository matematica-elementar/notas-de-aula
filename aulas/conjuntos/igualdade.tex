\section{Igualdade}

Na Seção~\ref{sec:intro}, vimos intuitivamente como dois conjuntos são iguais. Em seguida, veremos a definição de igualdade de conjuntos, isso é, a condição necessária e suficiente para que dois conjuntos sejam considerados iguais.

\begin{definition}[Igualdade de Conjuntos]
\label{def:=}
Sejam A e B conjuntos arbitrários, definimos:

	$$ A = B \text{ se e somente se } A \subset B \text{ e } B \subset A $$

\end{definition}

\begin{remark}
	Agora, em \nameref{inclusao:antissimetria}, a \textit{antissimetria} pode ser facilmente demonstrada; basta utilizar diretamente a definição de igualdade de conjuntos.
\end{remark}

% @TODO propriedades
% @TODO exemplos
% @TODO exercícios

\section{Igualdade}

Na Seção~\ref{section:intro} vimos intuitivamente como dois conjuntos são iguais. Em seguida veremos a definição de igualdade de conjuntos, isto é, a condição necessária e suficiente para que dois conjuntos sejam considerados iguais.

\begin{definition}
Sejam A e B conjuntos arbitrários, definimos:

	$$ A = B \text{ se e somente se } A \subset B \text{ e } B \subset A $$

\end{definition}

\begin{remark}
	Agora, nas \nameref{inclusao:antissimetria} a antissimetria pode ser facilmente demostrada utilizando diretamente a definição de igualdade de conjuntos.
\end{remark}

% @TODO propriedades
% @TODO exemplos
% @TODO exercícios

\section{Módulos}

% Exemplo 21
\begin{solution}
\begin{enumerate}
	\item[]
	\item Note que:
	%
	\begin{align*}
		\modu{2x-5} = 
		\begin{cases}
		2x-5,    & \text{se } 2x-5 \ge 0 \iff x \ge 5/2 \\
		-(2x-5), & \text{se } 2x-5 < 0 \iff x < 5/2
		\end{cases}
	\end{align*}
	%
	Se $x \ge 5/2$, teremos:
	%
	\begin{align*}
	\modu{2x-5}=3 &\iff 2x-5=3 \\
				  &\iff x=4
	\end{align*}
	%
	Como $x=4\ge5/2$, temos que $4 \in S$.

	Se $x<5/2$, teremos:
	%
	\begin{align*}
		\modu{2x-5}=3 &\iff -\prn{2x-5}=3 \\
					  &\iff -2x=2 \\
					  &\iff x=1 
	\end{align*}
	%
	Como $x=1<5/2$, então $1 \in S$.

	Das análises dos dois casos, concluímos que o conjunto solução é $S=\set{1,4}$.

	\item Observe que:
	%
	\begin{align*}
		\modu{2x-3}= 
		\begin{cases}
			2x-3,    & \text{se } 2x-3 \ge 0 \iff x \ge 3/2 \\
			-(2x-3), & \text{se } 2x-3 < 0 \iff x < 3/2
		\end{cases}
	\end{align*}
	%
	Se $x \ge 3/2$, teremos:
	%
	\begin{align*}
		\modu{2x-3}=1-3x &\iff 2x-3=1-3x
						 &\iff x = 4/5
	\end{align*}
	%
	Como $4/5 < 3/2$, então $4/5 \notin S$.

	Se $x < 3/2$, teremos:
	%
	\begin{align*}
		\modu{2x-3}=1-3x &\iff -\prn{2x-3}=1-3x
						 &\iff x = -2
	\end{align*}
	%
	Como $-2 < 3/2$, então $-2 \in S$.

	Das análises dos dois casos, concluímos que o conjunto solução é $S=\set{2}$.

	\item Note que:
	%
	\begin{align*}
		& \modu{3-x}= 
		\begin{cases}
			3-x,    & \text{se } 3-x \ge 0 \iff x \ne 3 \\
			-(3-x), & \text{se } 3-x < 0 \iff x > 3
		\end{cases}\\
		& \modu{x+1}= 
		\begin{cases}
			x+1,    & \text{se } x+1 \ge 0 \iff x \ge 3 \\
			-(x+1), & \text{se } x+1 < 0 \iff x < 3
		\end{cases}
	\end{align*}
	%
	\begin{itemize}
		\item Caso $x<-1$: 
		%
		\begin{align*}
			\modu{3-x}-\modu{x+1}=4 &\iff 3-x-\left[-(x+1)\right]=4\\
									&\iff 4=4 \ \ \ \text{ para todo } x \in \R
		\end{align*}
		%
		Temos, então, que $x<-1$ é solução para a equação.
		%
		\item Caso $-1 \le x \le 3$:
		%
		\begin{align*}
			\modu{3-x}-\modu{x+1}=4 &\iff 3-x-(x+1)=4\\
									&\iff -2x+2=4\\
									&\iff x=-1
		\end{align*}
		%
		Logo, $x=-1$ é solução da equação.
		%
		\item Caso $x>3$:
		%
		\begin{align*}
			\modu{3-x}-\modu{x+1}=4 &\iff -(3-x)-(x+1)=4\\
									&\iff -4=4
		\end{align*}
		%
		Nesse caso, não há soluções.
	\end{itemize}

	Das análises dos casos, conclui-se que $S=\set{x \in \R \tq x \le -1}$.
\end{enumerate}
\end{solution}

% Exemplo 23
\begin{solution}
Note que:
%
\begin{align*}
	& \modu{3-x}= 
	\begin{cases}
		3-x,    & \text{se } 3-x \ge 0 \iff x \le 3 \\
		-(3-x), & \text{se } 3-x < 0 \iff x > 3
	\end{cases}\\
	& \modu{x+1}= 
	\begin{cases}
		x+1,    & \text{se } x+1 \ge 0 \iff x \ge 1 \\
		-(x+1), & \text{se } x+1 < 0 \iff x < -1
	\end{cases}
\end{align*}

\begin{itemize}
	\item Caso $x<-1$:
	%
	\begin{align*}
		\modu{3-x}-\modu{x+1}\le4 & \iff 3-x-\left[-\prn{x+1}\right]\le 4\\
			-					  & \iff 4 \le 4 \ \ \ \ \text{para todo } x \in \R
	\end{align*}
	%
	Tem-se, então, que $x<-1$ é solução para a inequação.
\end{itemize}

\end{solution}
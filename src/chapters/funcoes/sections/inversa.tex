\section{Função inversa}

\begin{definition}
\label{def:funcao-inversa}
Uma função $f: X \to Y$ é \emph{invertível} se existe uma função $g: Y \to X$ tal que:
%
\begin{enumerate}[(i)]
  \item $f \circ g = \identity Y$;
  \item $g \circ f = \identity X$.
\end{enumerate}
%
Nesse caso, a função $g$ é dita \emph{função inversa} de $f$ e denotada por $g = f^{-1}$.
\end{definition}

\begin{example}
\label{ex:q-inversa-de-p}
	A função $q$ é inversa de $p$?
\end{example}

\begin{solution}
Do Exemplo \ref{ex:comp-pq}, já temos que $p \circ q = \identity{\R_+}$.
Então, devemos verificar se $q \circ p = \identity{\R}$.
Para tal, seja $x \in \R$. 
Note que:
%
\begin{align*}
	\prn{q \circ p}(x) &= q(p(x)) \\ &= q(x^2) \\ &= \sqrt{x^2} \\ &= \modu{x}.
\end{align*}
%
Assim, temos que $q \circ p \ne \identity{\R}$, pois, por exemplo, $\identity{\R}(-3) = -3 \ne 3 = \modu{-3} = \prn{q \circ p}(-3)$.
Portanto, $q$ não é a função inversa de $p$.
\end{solution}

O Exemplo \ref{ex:q-inversa-de-p} ilustra a importância de se verificar, quando se quer provar que uma função é inversa da outra, todas as condições da definição.
Demonstrar a validade de apenas uma delas não garante que uma função é inversa de outra, mesmo que, inicialmente, pensemos o contrário.
Além disso, junto do exemplo \ref{ex:comp-pq}, obtemos que a composição de funções NÃO é uma operação comutativa, já que $p \circ q  \ne q \circ p$.

\begin{onlineact}
	\khan{https://pt.khanacademy.org/math/algebra2/manipulating-functions/verifying-that-functions-are-inverses/e/inverses_of_functions}{Verifique Funções Inversas}.
\end{onlineact}
\subsubsection{Translação}

\begin{example}
    Compare os gráficos das funções reais $f, g , h: \R \to \R$ tais que
$f(x) = \sen x$, \\ $g(x) = f(x) + 1 = \sen x +1$ , \\ $h(x)=
f(x+\frac {\pi} 2)= \sen (x+ \frac {\pi} 2)$.
\end{example}

Dessa forma, se a função real $g$ é tal que $g(x) = f(x+b) +a$,
então o gráfico de $g$ pode ser obtido, do gráfico de $f$, através
de uma translação horizontal determinada pelo parâmetro $b$, e uma
translação vertical determinada pelo parâmetro $a$. 
\begin{itemize}
  \item O translado vertical será:
        \begin{itemize}
          \item No sentido positivo do eixo $y$ (para cima), se
          $a>0$;
          \item No sentido negativo do eixo $y$ (para baixo), se
          $a<0$.
        \end{itemize} 
  \item O translado horizontal será:
        \begin{itemize}
          \item No sentido positivo do eixo $x$ (para a direita), se $b<0$;
          \item No sentido negativo do eixo $x$ (para a esquerda), se $b>0$.
        \end{itemize}
\end{itemize}
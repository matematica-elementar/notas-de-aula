\subsection{Funções Limitadas}

\begin{definition}
    Seja $f: D \subset \R \to \R$ uma função.
\begin{enumerate}[(i)]
  \item $f$ é \textdef{limitada superiormente} se existe $M \in \R$ tal
  que $f(x) \leq M$ para todo $x \in D$;
  \item $f$ é \textdef{limitada inferiormente} se existe $M \in \R$ tal
  que $f(x) \geq M$ para todo $x \in D$;
  \item $x_0 \in D$ é um \textdef{ponto de máximo absoluto} de $f$ se
  $f(x_0) \geq f(x)$ para todo $x \in D$;
  \item $x_0 \in D$ é um \textdef{ponto de mínimo absoluto} de $f$ se
  $f(x_0) \leq f(x)$ para todo $x \in D$;
  \item $x_0 \in D$ é um \textdef{ponto de máximo local} de $f$ se
  existe $r>0$ tal que $f(x_0) \geq f(x)$ para todo $x \in D \cap \paren{x_0 - r , x_0+r}$;
  \item $x_0 \in D$ é um \textdef{ponto de mínimo local} de $f$ se
  existe $r>0$ tal que $f(x_0) \leq f(x)$ para todo $x \in D \cap \paren{x_0 - r ,
  x_0+r}$.
\end{enumerate}
\end{definition}

\begin{example}
    A função $h : \left( -1 ; 6 \right] \to \R$, cujo gráfico é esboçado
na Imagem~\ref{img:funcao-h-descontinua}, é definida por:
%
$$h(x) = \begin{cases}
                                3x-x^2 & \ \text{ se } \ x\leq 2 \\
                                \modu{x-4} +1 & \ \text{ se } \ 2 < x \leq 5 \\
                                2 & \ \text{ se } \ x > 5 \\
                                \end{cases}.$$
%
\begin{figure}
\centering
\includegraphics[scale=0.5]{\imgdirfromsection/funcao-h-descontinua}
\caption{Gráfico da função $h$.}
\label{img:funcao-h-descontinua}
\end{figure}
%
Identifique, pelo gráfico, os intervalos de monotonicidade e os extremos locais e absolutos $h$.
\end{example}

\begin{solution}
    A partir do gráfico, pode-se perceber que a função $h$ é crescente nos intervalos $(-1;1{,}5]$ e $[4;5]$,
    decrescente em $[1,5; 2]$ e $(2;4]$ e constante em $[5;6]$.     
    Também por observação do gráfico da função, conclui-se que o ponto $(1{,}5; 2{,}25)$ é um mínimo local, e o ponto $(4,1)$ é um máximo local.
    A função não possui extremos absolutos.
    Algumas observações, baseadas na análise do gráfico, são importantes acerca da solução apresentada acima.
    \begin{itemize}
        \item $-1$ não está presente no intervalo de crescimento $(-1;1{,}5]$ por não ser elemento do domínio da função;
        \item $1{,}5$ deve estar no intervalo de crescimento $(-1;1{,}5]$. Note que, quaisquer $x_1, x_2 \in (-1;1{,}5]$ é satisfeito que $x_1 < x_2$ implica $f(x_1) < f(x_2)$, ou seja, a condição para que a função seja crescente nesse intervalo. Em particular, como $1{,}5$ é o maior dos valores de $(-1;1{,}5]$, note que é válido que $x_1 < 1{,}5$ implica $f(x_1) < f(1{,}5)$;
        \item Analogamente ao ponto anterior, é necessário que $1{,}5$ e $2$ estejam no intervalo de decrescimento $[1,5; 2]$. Assim como todos os demais extremos dos intervalos fechados.
        \item $2$ não deve fazer parte do intervalo de decrescimento $(2; 4]$, pois há uma bola aberta nesse intervalo de decrescimento. Se $2$ pertencesse a tal intervalo, teríamos devido à definição de função decrescente,que $2<2{,}1$ implicaria em $2 = f(2) < f(2{,}1 = 2{,}9$. Essa implicação claramente é falsa;
        \item O fato há um mínimo local quando $x = 4$, pode ser observado porque $4$ é maior valor de um intervalo de decrescimento e o menor de um intervalo de crescimento. Em outras palavras, em $x=4$, a função que estava decrescendo em $(2;4]$ passa a crescer em $[4;5]$, atingindo o menor valor localmente nessa mudança de comportamento da monotonicidade. É comum que mudanças da monotonicidade da função determinem extremos locais.
        \item Analogamente ao ponto anterior, justifica-se que em $x=1{,}5$ tenhamos um máximo local.
        %\item extremos no intervalo de constância
        %\item demonstrar usando os intervalos de monotonicidade que 4 é mínimo local
    \end{itemize}
\end{solution}

\begin{onlineact}
    \khan{https://pt.khanacademy.org/math/algebra/algebra-functions/positive-negative-increasing-decreasing-intervals/e/increasing-decreasing-intervals-of-functions}
    {Intervalos Crescentes e Decrescentes}.
\end{onlineact}

\begin{onlineact}
    \khan{https://pt.khanacademy.org/math/algebra/algebra-functions/maximum-and-minimum-points/e/recognize-maxima-and-minima}
    {Mínimos e Máximos Relativos}.
\end{onlineact}

\begin{onlineact}
    \khan{https://pt.khanacademy.org/math/algebra/algebra-functions/maximum-and-minimum-points/e/recognize-absolute-maxima-and-minima}
    {Mínimos e Máximos Absolutos}.
\end{onlineact}
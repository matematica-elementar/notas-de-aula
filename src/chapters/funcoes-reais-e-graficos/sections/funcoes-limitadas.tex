\subsection{Funções Limitadas}

\begin{definition}
    Seja $f: D \subset \R \to \R$ uma função.
\begin{enumerate}[(i)]
  \item $f$ é \emph{limitada superiormente} se existe $M \in \R$ tal
  que $f(x) \leq M$ para todo $x \in D$;
  \item $f$ é \emph{limitada inferiormente} se existe $M \in \R$ tal
  que $f(x) \geq M$ para todo $x \in D$;
  \item $x_0 \in D$ é um \emph{ponto de máximo absoluto} de $f$ se
  $f(x_0) \geq f(x)$ para todo $x \in D$;
  \item $x_0 \in D$ é um \emph{ponto de mínimo absoluto} de $f$ se
  $f(x_0) \leq f(x)$ para todo $x \in D$;
  \item $x_0 \in D$ é um \emph{ponto de máximo local} de $f$ se
  existe $r>0$ tal que $f(x_0) \geq f(x)$ para todo $x \in D \inter \paren{x_0 - r , x_0+r}$;
  \item $x_0 \in D$ é um \emph{ponto de mínimo local} de $f$ se
  existe $r>0$ tal que $f(x_0) \leq f(x)$ para todo $x \in D \inter \paren{x_0 - r ,
  x_0+r}$.
\end{enumerate}
\end{definition}

\begin{example}
    A função $h : \left( -1 ; 6 \right] \to \R$, cujo gráfico é esboçado
na Imagem~\ref{img:funcao-limitada-descontinua}, é definida por:
%
$$h(x) = \begin{cases}
                                3x-x^2 & \ \text{ se } \ x\leq 2 \\
                                \modu{x-4} +1 & \ \text{ se } \ 2 < x \leq 5 \\
                                2 & \ \text{ se } \ x > 5 \\
                                \end{cases}.$$
%
\begin{figure}[H]
    \centering
    \includegraphics[scale=0.3]{\imgdirfromsection/funcao-limitada-descontinua.png}
    \caption{Gráfico da função $h$.}
    \label{img:funcao-limitada-descontinua}
\end{figure}
%
Identifique, pelo gráfico, os intervalos de monotonicidade e os extremos locais e absolutos de $h$.
\end{example}

\begin{solution}
    A partir do gráfico, pode-se perceber que a função $h$ é crescente nos intervalos $(-1;1{,}5]$ e $[4;5]$,
    decrescente em $[1,5; 2]$ e $(2;4]$ e constante em $[5;6]$.     
    Também por observação do gráfico da função, conclui-se que o ponto $(1{,}5; 2{,}25)$ é um mínimo local,
    e o ponto $(4,1)$ é um máximo local. 
    Além disso, todos os pontos tais que $x \in [5;6]$ são máximos locais,
    e todos os pontos tais que $x \in (5; 6]$ são mínimos locais.
    A função não possui extremos absolutos.

    Na solução dada, alguns fatos podem causar certa confusão. 
    Assim, serão feitas, a seguir, algumas observações baseadas na análise do gráfico para elucidá-los.
    %
    \begin{enumerate}[(a)]
        \item $-1$ não está presente no intervalo de crescimento $(-1;1{,}5]$ por não pertencer ao domínio da função;
        \item \label{item:intervalo-crescimento} $1{,}5$ deve estar no intervalo de crescimento $(-1;1{,}5]$. 
        Note que, para quaisquer $x_1, x_2 \in (-1;1{,}5]$, é verdade que $x_1 < x_2$ implica $h(x_1) < h(x_2)$,
        uma vez que a função é crescente nesse intervalo. 
        Em particular, como $1{,}5$ é o maior dos valores de $(-1;1{,}5]$, 
        temos que $x_1 < 1{,}5$ implica $h(x_1) < h(1{,}5)$;
        \item Analogamente ao item \ref{item:intervalo-crescimento},
        é necessário que $1{,}5$ e $2$ estejam no intervalo de decrescimento $[1,5; 2]$.
        A análise é similar para os extremos dos outros intervalos fechados;
        \item $2$ não deve fazer parte do intervalo de decrescimento $(2; 4]$, 
        pois há uma bola aberta nesse intervalo. 
        Se $2$ pertencesse ao intervalo, teríamos, devido à definição de função decrescente,
        que $2<2{,}1$ implicaria em $2 = h(2) < h(2{,}1) = 2{,}9$. 
        Essa implicação claramente é falsa;
        \item \label{item:minimo-local} O fato de que há um mínimo local quando $x = 4$ se dá pois $4$ é o maior valor de um intervalo de decrescimento e o menor de um intervalo de crescimento. 
        Em outras palavras, em $x=4$, a função que estava decrescendo em $(2;4]$ passa a crescer em $[4;5]$,
        atingindo o menor valor localmente nessa mudança de comportamento da monotonicidade. 
        É comum que mudanças da monotonicidade da função determinem extremos locais.
        No item \ref{item:minimo-local-formal}, é demonstrado que, de fato, há um mínimo local em $x=4$.
        \item A justificativa para a existência de um máximo local em $x=1{,}5$ é análoga à do item \ref{item:minimo-local};
        \item Em todo o intervalo $[5; 6]$, em que a função é constante, 
        os pontos são máximos locais e mínimos locais, exceto pelo fato de que não há mínimo local em $x=5$. 
        Para qualquer $x_0\in (5;6]$, existe um intervalo $(x_0 - r; x_0+r)$ em torno de $x_0$ tal que $f(x)= f(x_0)$ para todo $x \in (x_0 - r; x_0+r) \inter D$, em que $D = \left(-1,6\right]$.
        Note que $f(x)= f(x_0)$ implica tanto $f(x) \leq f(x_0)$ quanto $f(x_0)\leq f(x)$, 
        o que justifica a existência dos mínimos e máximos locais em $(5;6]$. 
        Quando $x_0=5$, conseguimos concluir que $f(x) \le f(x_0)$ mas não $f(x_0) \le f(x)$, 
        pois em qualquer intervalo $(5 - r; 5+r)$ existe $x < 5$ tal que $f(x) < f(5)$.
        Assim, em $x_0=5$ há máximo local mas não mínimo local;
        \item \label{item:minimo-local-formal} Tendo, como verdade, que a função é decrescente em $(2; 4]$ e crescente em $[4; 5]$, 
        provemos que ocorre um mínimo local em $x=4$. Para tanto, precisamos mostrar que existe $r>0$ tal que, 
        para todo $x \in (4-r; 4+r)$, vale $h(4) \leq h(x)$. 
        Tome $r=1$, e seja $x \in (4-1; 4+1) = (3; 5)$. 
        Separaremos, em três casos possíveis para $x$, a verificação de que $h(4) \leq h(x)$. 
        Caso $x \in (3; 4)$, note que $x, 4 \in (2; 4]$, um intervalo de decrescimento da função. 
        Assim, como $x<4$, então $h(4)< h(x)$, o que implica que $h(4) \leq h(x)$. 
        Agora, caso $x \in (4; 5)$, note que $x, 4 \in [4; 5]$, um intervalo de crescimento da função. 
        Assim, como $4<x$, então $h(4)< h(x)$, e, consequentemente, $h(4)\le h(x)$.
        Finalmente, caso $x=4$, é óbvio que $h(4)=h(x)$, o que também implica que $h(4)\le h(x)$. 
        Portanto, em $x=4$ há um mínimo local para a função $h$.
    \end{enumerate}
\end{solution}

\begin{onlineact}
    \khan{https://pt.khanacademy.org/math/algebra/algebra-functions/positive-negative-increasing-decreasing-intervals/e/increasing-decreasing-intervals-of-functions}
    {Intervalos Crescentes e Decrescentes}.
\end{onlineact}

\begin{onlineact}
    \khan{https://pt.khanacademy.org/math/algebra/algebra-functions/maximum-and-minimum-points/e/recognize-maxima-and-minima}
    {Mínimos e Máximos Relativos}.
\end{onlineact}

\begin{onlineact}
    \khan{https://pt.khanacademy.org/math/algebra/algebra-functions/maximum-and-minimum-points/e/recognize-absolute-maxima-and-minima}
    {Mínimos e Máximos Absolutos}.
\end{onlineact}
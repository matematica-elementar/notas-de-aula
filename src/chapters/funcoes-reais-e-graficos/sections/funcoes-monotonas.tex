\subsection{Funções Monótonas}

\begin{definition}
\label{def:funcao-monotona}
    Seja $f: D \subset \R \to \R$ uma função. Dizemos que
\begin{enumerate}[(i)]
  \item $f$ é \textdef{monótona (estritamente) crescente} se, para todos $x_1, x_2 \in D$,
  $$x_1 < x_2 \implies f(x_1) < f(x_2);$$
  \item $f$ é \textdef{monótona não decrescente} se, para todos $x_1, x_2 \in D$,
  $$x_1 < x_2 \implies f(x_1) \leq f(x_2);$$
  \item $f$ é \textdef{monótona (estritamente) decrescente} se, para todos $x_1, x_2 \in D$,
  $$x_1 < x_2 \implies f(x_1) > f(x_2);$$
  \item $f$ é \textdef{monótona não crescente} se, para todos $x_1, x_2 \in D$,
  $$x_1 < x_2 \implies f(x_1) \geq f(x_2).$$
\end{enumerate}
\end{definition}

Nas mesmas condições da Definição \ref{def:funcao-monotona}, se $f(x) = k \in
\R$ para todo $x \in D$, dizemos que $f$ é \sub{constante}.
Se $I \subset D$ é um intervalo, definimos a monotonicidade de $f$
no intervalo $I$ de maneira análoga ao feito anteriormente. Por
exemplo: \\
$f$ é \sub{monótona (estritamente) crescente em $I$} se, para todos
$x_1, x_2 \in I$,
  $$x_1 < x_2 \implies f(x_1) < f(x_2).$$
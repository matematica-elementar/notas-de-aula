\subsubsection{Função Inversa da Tangente}

\begin{example}
Como $tan: \paren{- \frac {\pi} 2, \frac {\pi} 2} \to \R$ é
bijetiva, então essa função possui inversa, que chamamos de
\sub{arco tangente} e denotamos por $\arctan : \R \to
\paren{- \frac {\pi} 2, \frac {\pi} 2}$. Seu gráfico é
\begin{figure}
\centering
\includegraphics[width=5.5cm]{\imgdirfromsection/grafarctan.jpg}
\end{figure}
\end{example}

\begin{onlineact}
    \khan{https://pt.khanacademy.org/math/trigonometry/trigonometry-right-triangles/reciprocal-trig-ratios/e/reciprocal_trig_funcs}
    {Razões Trigonométricas Recíprocas}.
\end{onlineact}

\begin{onlineact}
    \khan{https://pt.khanacademy.org/math/trigonometry/trigonometry-right-triangles/modeling-with-right-triangles/e/applying-right-triangles}
    {Problemas com Triângulos Retângulos}.
\end{onlineact}
\section{Introdução}

A trigonometria é estudada desde os gregos, e sua motivação inicial
era determinar os seis elementos principais do triângulo --- seus lados
e ângulos --- quando conhecidos alguns deles. Com a criação do Cálculo Infinitesimal, veio a necessidade da criação
de funções trigonométricas definidas em $\reais$, conforme será visto neste capítulo.

As funções trigonométricas referidas anteriormente ganharam notoriedade quando, em 1822, Joseph Fourier
provou que toda função periódica é uma soma --- finita ou infinita --- de
funções do tipo $a\cos (nx) +b \sen (nx)$. Tal descoberta deu origem
à área da Matemática chamada Análise de Fourier. Além disso,
segundo o banco de dados da revista ``Mathematical Reviews'', o nome
mais citado nos títulos de trabalhos matemáticos nos últimos 50 anos
é o de Fourier.

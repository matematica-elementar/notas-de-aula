\section{Trigonometria no Triângulo Retângulo}

\begin{definition}
\label{def:seno-cosseno-triangulo-retangulo}
Em um triângulo retângulo $ABC$, como na Imagem~\ref{fig:triangulo-retangulo}, definem-se o
\emph{cosseno} ($\cos$) e o \emph{seno} ($\sen$) dos ângulos agudos do
triângulo:
%
$$\cos \widehat B = \frac c a = \frac {\text{cateto
adjacente}}{\text{hipotenusa}}, \ \ \ \ \sen \widehat B = \frac b a = \frac
{\text{cateto oposto}}{\text{hipotenusa}},$$
$$\cos \widehat C = \frac b a \ \ \ \ \text{e} \ \ \ \ \sen \widehat
C = \frac c a.$$    
%
\begin{figure}[H]
\centering
\includegraphics[scale=0.4]{\imgdirfromsection/triangret.jpg}
\caption{Triângulo retângulo qualquer.}
\label{fig:triangulo-retangulo}
\end{figure}
\end{definition}

\begin{remark}
As relações definidas como o foram na Definição~\ref{def:seno-cosseno-triangulo-retangulo} são únicas para cada ângulo em
decorrência da proporcionalidade dos lados de triângulos
semelhantes. Portanto, calculam-se o seno e o cosseno de um ângulo
independentemente do triângulo retângulo que o contém.
\end{remark}

\begin{proposition}
Os seguintes valem:
\begin{itemize}
	\item O cosseno de um ângulo agudo é igual ao seno do seu
	complementar e vice-versa (daí surgiu o termo ``cosseno'': seno do complemento);
	\item O seno e o cosseno são números compreendidos entre 0 e 1, uma vez que são razões entre um cateto 
	e a hipotenusa de um triângulo retângulo.
\end{itemize}
\end{proposition}

\begin{proof}
	Sejam $\widehat B$ um ângulo agudo e $\triangle ABC$ um triângulo tal que um de seus ângulos mede $\widehat B$.
	A situação é ilustrada na Imagem~\ref{img:triangulo-retangulo-generico}.
	%
	\begin{figure}[H]
		\centering
		\includegraphics[scale=0.65]{\imgdirfromsection/triangulo-retangulo-generico.jpg}
		\caption{Triângulo retângulo qualquer.}
		\label{img:triangulo-retangulo-generico}
	\end{figure}
	%
	Temos que $\widehat A + \widehat B  + \widehat C = 180\degree$, ou seja, $\widehat B  + \widehat C = 90\degree$.
	Logo, $\widehat B$ e $\widehat C$ são complementares. Segue que:
	%
	$$\cos \widehat B = \frac c a = \sen \widehat C.$$ Portanto, o cosseno de um ângulo agudo é igual ao seno do seu complementar,
	e vice-versa. Ademais, pelo fato de a hipotenusa ser o maior dos lados de um triângulo retângulo, já que é oposta ao 
	maior dos ângulos desse triângulo, segue que qualquer razão de um cateto pela hipotenusa será um número entre 0 e 1.
\end{proof}

\begin{proposition}[Relação Fundamental da Trigonometria]
Seja $\widehat B$ um dos ângulos agudos de um triângulo retângulo. Então:
$$\sen^2 \widehat B + \cos^2 \widehat B = 1.$$
\end{proposition}

\begin{proof}
	Seja $\triangle ABC$ um triângulo retângulo tal que $\widehat B$ é um de seus ângulos agudos, sua hipotenusa mede $a$ e os catetos, $b$ e $c$. A situação é mostrada na Imagem~\ref{img:prova-relacao-fundamental-trigonometria}.
	%
	\begin{figure}[H]
		\centering
		\includegraphics[scale=0.65]{\imgdirfromsection/triangulo-retangulo-generico2.jpg}
		\caption{Triângulo retângulo qualquer.}
		\label{img:prova-relacao-fundamental-trigonometria}
	\end{figure}
	%
	Do Teorema de Pitágoras, temos:
	%
	\begin{align*}
		b^2 + c^2 = a^2 &\iff \frac{b^2}{a^2} + \frac{c^2}{a^2} = 1 \\ &\iff \prn{\frac b a}^2 + \prn{\frac c a}^2 = 1 \\ 
		&\iff \sen^2 \widehat B  +  \cos^2 \widehat B= 1.
	\end{align*}
\end{proof}

\begin{onlineact}
	\khan{https://pt.khanacademy.org/math/trigonometry/trigonometry-right-triangles/intro-to-the-trig-ratios/e/trigonometry_1}{Razões Trigonométricas em Triângulos Retângulos}.
\end{onlineact}

\begin{onlineact}
	\khan{https://pt.khanacademy.org/math/trigonometry/trigonometry-right-triangles/trig-solve-for-a-side/e/trigonometry_2}{Como Calcular a Medida de um Lado em Triângulos Retângulos}.
\end{onlineact}
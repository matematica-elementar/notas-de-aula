\section{Exercícios}

\begin{exercise}
    Na Definição~\ref{def:seno-cosseno-triangulo-retangulo}, definimos seno e cosseno de um ângulo no
triângulo retângulo. Como você definiria, com os lados de um
triângulo retângulo, as demais relações trigonométricas da Definição
\ref{def:outras-funcoes-trigonometricas}?
\end{exercise}

\begin{exercise}
Saber para quais valores $t$ são válidas algumas equações
envolvendo equações trigonométricas é muito importante. Determine o
conjunto solução de cada uma das equações abaixo:
\begin{enumerate}[(a)]
  \item $\sen t = 0$, $\cos t = 0$ e $\tan t = 0$;
  \item $\sen t = 1$, $\cos t = 1$;
  \item $\sen t = -1$, $\cos t = -1$ e $\tan t = -1$;
  \item $\sen t = \cos t$ e $\tan t = 1$;
  \item $\csc t = 0$, $\sec t = 0$ e $\cot t = 0$;
  \item $\csc t = 1$, $\sec t = 1$;
  \item $\csc t = -1$, $\sec t = -1$ e $\cot t = -1$;
  \item $\csc t = \sec t$ e $\cot t = 1$.
\end{enumerate}
\end{exercise}

\begin{exercise}
    A figura abaixo representa o gráfico da função $f_1 : \R \to
\R$, $f_1(x) = x \cdot \sen x$, traçado no intervalo $\bracket{-20 \pi,
20 \pi}$, juntamente com as retas $y=x$ e $y=-x$.
\begin{center}
\includegraphics[scale=0.35]{\imgdirfromsection/grafxsenx.jpg}
\end{center}
\begin{enumerate}[(a)]
  \item Explique por que o gráfico de $f_1$ fica limitado entre
  essas retas e indique todos os pontos em que o gráfico toca as retas;
  \item Considere a seguinte afirmação: \emph{Os máximos e
  mínimos locais da função $f_1$ ocorrem nos mesmos valores
  de $x$ que os da função seno.} Esta afirmação é verdadeira?
  \item Como você espera visualizar o gráfico da função $f_2: \R \to
  \R$, definida por $f_2(x) = x^2 \cdot \sen x$?
\end{enumerate}
\end{exercise}

\begin{exercise}
    Na figura abaixo, os segmentos $AD$ e $OD$ representam,
respectivamente, $\tan x$ e $\sec x$.
\begin{center}
\includegraphics[scale=0.5]{\imgdirfromsection/circtansec.jpg}
\end{center}
\begin{enumerate}[(a)]
  \item Justifique a afirmação acima;
  \item Qual a interpretação dos sinais de $\tan x$ e $\sec x$ na
  figura?
  \item Faça uma figura análoga para representar $\cot x$ e $\csc
  x$, justificando a construção.
\end{enumerate}
\end{exercise}

\begin{exercise}
    Encontre as três menores soluções positivas da equação $$\cos
\paren{3x - \frac {\pi} 4} = 0.$$
\end{exercise}

\begin{exercise}
    Mostre que o perímetro do pentágono regular inscrito em um
círculo unitário é dado por $10\sen \frac {\pi} 5$.
\end{exercise}

\begin{exercise}
    Considere a função $f: \R \to \R$ definida por $f(x) = \sen
\paren{ax}+\sen \paren{bx}$, em que $a$ e $b$ são constantes reais.
\begin{enumerate}[(a)]
  \item Mostre que, se $a$ e $b$ são racionais, então $f$ é
  periódica;\\
  \emph{Dica:} Mostre que o período de $\sen \paren{ax}$ é $\frac
  {2\pi} a$.
  \item A recíproca da afirmação do item anterior é verdadeira?
  Justifique sua resposta.
\end{enumerate}
\end{exercise}

\begin{exercise}
    Prove as identidades abaixo, válidas para todo $x$ onde as
expressões estão definidas:
\begin{enumerate}[(a)]
  \item $\frac{1-\tan^2 x}{1+\tan^2 x} = 1 - 2\sen^2 x$;
  \item $\frac{\cos x - \sen x}{\cos x + \sen x} = \frac{1 - \tan x}{1+\tan
  x}$;
  \item $\frac{\sen x}{\csc x - \cot x} = 1+ \cos x$;
  \item $\cos^2 x = \frac {1+\cos \paren{2x}} 2$;
  \item $\sen^2 x = \frac {1-\cos \paren{2x}} 2$;
  \item $\frac{1-\tan^2 x}{1+\tan^2 x} = \cos^2 x - \sen^2 x = \cos \paren{2x}$;
  \item $\frac{2\tan x}{1+\tan^2 x} = 2\sen x \cos x= \sen\paren{2x}$.
\end{enumerate}
\end{exercise}

\begin{exercise}
    Use as fórmulas de seno e cosseno da soma para determinar os
senos e cossenos dos seguintes ângulos (medidos em radianos): $\frac
{\pi} 8$, $\frac{\pi} {12}$, $\frac {3\pi} 8$ e $\frac{5\pi}{12}$.
\end{exercise}

\begin{exercise}
    Obtenha fórmulas para $\tan\paren{\alpha + \beta}$ e para
$\sec\paren{\alpha + \beta}$, em função de $\tan \alpha$ e $\tan
\beta$.
\end{exercise}

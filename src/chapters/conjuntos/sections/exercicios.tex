\section{Exercícios}

\begin{exercise}
  \label{exe:vazio-notacao}
	De que formas podemos representar o conjunto vazio utilizando as duas notações de conjuntos apresentadas na Seção \ref{sec:intro}?
\end{exercise}

\begin{exercise}
  \label{exe:vazios-tricky}
	Decida quais das afirmações a seguir estão corretas. Justifique suas respostas.
	\begin{enumerate}[a)]
		\item $\vazio \pertence \vazio$;
		\item $\vazio \contido \vazio$;
		\item $\vazio \pertence \unitario{\vazio}$;
		\item $\vazio \contido \unitario{\vazio}$.
	\end{enumerate}
\end{exercise}

\begin{exercise}
  Demonstre que os seguintes itens são equivalentes:
    \begin{enumerate}[a)]
      \item $A \uniao B = B$;
      \item $A \contido B$;
      \item $A \inter B = A$;
    \end{enumerate}
    \begin{hint}
      Para tanto, é preciso provar que a) $\sse$ b) e b) $\sse$ c). Outra maneira é provar a) $\implica$ b), b) $\implica$ c) e por fim, c) $\implica$ a).
    \end{hint}
\end{exercise}

\begin{exercise}
  O diagrama de Venn para os conjuntos $X$, $Y$, $Z$ decompõe o plano em oito regiões. Numere essas regiões e exprima cada um dos conjuntos abaixo como reunião de algumas dessas regiões. (Por exemplo: $X \inter Y = 1 \uniao 2$.)
  \begin{enumerate}[a)]
    \item $(X^C \uniao Y)^C$;
    \item $(X^C \uniao Y) \uniao Z^C$;
    \item $(X^C \inter Y) \uniao (X \inter Z^C)$;
    \item $(X \uniao Y)^C \inter Z$.
  \end{enumerate}
\end{exercise}

\begin{exercise}
  Exprimindo cada membro como reunião de regiões numeradas, prove as igualdades:
  \begin{enumerate}[a)]
    \item $(X \uniao Y) \inter Z = (X \inter Z) \uniao (Y \inter Z)$;
    \item $X \uniao (Y \inter Z )^C = X \uniao Y^C \uniao Z^C$.
  \end{enumerate}
\end{exercise}

\begin{exercise}
  Sejam $A$, $B$ e $C$ conjuntos. Mostre que 
  $$A \inter B = B \inter C \sse (A \uniao B) \inter C =(A \inter B) \uniao (A \inter C). $$
\end{exercise}

\begin{exercise}
  Sejam $A$, $B$ e $C$ conjuntos. Determine uma condição necessária e suficiente para que se tenha 
  \[
    A \uniao (B \inter C) = (A \uniao B) \inter C.
  \]
\end{exercise}

\begin{exercise}
  Considere $A$, $A\linha$, $B$ e $B\linha$ conjuntos tais que $A  \subset A\linha$ e 
  $B \subset B\linha$. Prove que, se $A\linha \inter B\linha=\vazio$, então
  $A\inter B = \vazio$.  
\end{exercise}

\begin{exercise}
  Complete as demonstrações em \nameref{prop:uniao-e-intersecao} que não foram feitas em sala de aula.
\end{exercise}

\begin{exercise}
  Recorde a definição da diferença entre conjuntos:
  \[
    B \menos A = \conjunto{ x \tq x \pertence B \e x \naopertence A}.
  \]
  Dados conjuntos $A$, $B$ e $C$ arbitrários, mostre que
  \begin{enumerate}[a)]
    \item $B \menos A = B \inter \compl A$;
    \item $(B \menos A) \uniao A = B \uniao A$;
    \item $\compl{(B \menos A)} = B\complementar \uniao A$;
    \item $A \uniao \paren{ B \menos C } = \paren{A \uniao B } \menos \paren{C \menos A }$;
    \item $A \inter A^C = \vazio$;
    \item $A \menos \paren{B \inter C } = \paren{A \menos B } \uniao \paren{A \menos C }$;
    \item $A \menos \paren{ A \inter B } = A \menos B$;
    \item $B \menos A = \vazio$ se, e somente se, $B \contido A$;
    \item $B \menos A = B$ se, e somente se, $A \inter B = \vazio$;
    \item $B \menos A = A \menos B$ se, e somente se, $A = B$.
  \end{enumerate}
\end{exercise}

\begin{exercise}
  Sejam $A$, $B$ e $C$ conjuntos.
        
       Para cada uma das sentenças abaixo, diga se ela é Verdadeira ou Falsa. No caso de Verdadeira, demonstre-a, ou refute com um contra-exemplo para o caso de ser Falsa.
       \begin{enumerate}[a)]
            \item $A \setminus ( B \cap C) = (A \setminus B) \cap (A \setminus C)$;
            \item $(A \cap B) \setminus C = (A \setminus C) \cap (B \setminus C)$;
            \item $(A \cup B) \setminus C = (A \setminus C) \cup (B \setminus C)$;
            \item $A \setminus ( B \cup C) = (A \setminus B) \cup (A \setminus C)$;
            \item Se $A \cap B = \emptyset$ e $A \cap C = \emptyset$, então $A \setminus ( B \cup C) = (A \setminus B) \cup (A \setminus C)$.
       \end{enumerate}
\end{exercise}

\begin{exercise}
  Em um fórum de dúvidas sobre o jogo de LoL, os desenvolvedores explicaram como escolhem o campeão de um jogador quando este solicita a troca do campeão no modo ARAM. O novo campeão é escolhido aleatoriamente em um conjunto formado por $$\left(A \setminus B \right) \setminus C.$$
  O conjunto $A$ é o conjunto dos campeões do jogo, $B$ representa o conjunto dos campeões que já estão sendo utilizados por outros jogadores e $C$ é formado pelos campeões proibidos para aquela partida por qualquer que seja o motivo.  \begin{enumerate}[a)]
    \item Mostre que $$\left(A \setminus B \right) \setminus C = A \setminus \left( B \cup C \right).$$ Conclua se há ou não a possibilidade de aparecer um campeão que esteja sendo utilizado por outro jogador no momento da troca. 
    \item Se o conjunto inicial fosse substituído por $$ A \setminus \left( B \setminus C \right),$$ o resultado seria o mesmo? Justifique.
  \end{enumerate}
\end{exercise}

\begin{exercise}
  Considere $A$, $B$ e $C$ conjuntos quaisquer. Determine uma condição necessária e suficiente para que se tenha
  $$A \setminus \paren{B \setminus C} = \paren{A \setminus B} \setminus C.$$
\end{exercise}

\begin{exercise}
  Dados dois conjuntos $A$ e $B$, define-se a \emph{diferença simétrica} de $A$ por $B$, denotada por $A \bigtriangleup B$, como sendo o conjunto formado por todos os elementos que estão \emph{exatamente} em um dos dois conjuntos. Isto é,
$$A \bigtriangleup B = \left( A \menos B \right) \cup \left( B \menos A \right).$$

Mostre que $A \bigtriangleup B = A$ se, e somente se, $B = \vazio$.
\end{exercise}

\begin{exercise}
  Dados $A$ e $B$ conjuntos quaisquer, analise as afirmações abaixo. Caso seja verdadeira, prove-a. Caso ela seja falsa, dê exemplos de conjuntos e utilize-os para demonstrar que a afirmativa nem sempre ocorre:

  \begin{enumerate}[a)]
    \item Se $A \triangle B = A^C \cup B^C$, então $A = \mathcal U$;
    \item A recíproca da afirmação do item anterior é válida.
  \end{enumerate}
\end{exercise}

\begin{exercise}
  Dê exemplos de implicações, envolvendo conteúdos de ensino médio, que sejam: verdadeiras com recíproca verdadeira; verdadeiras com recíproca falsa; falsas, com recíproca verdadeira; falsas, com recíproca falsa.
\end{exercise}

\begin{exercise}
  Considere $P$, $Q$ e $R$ condições aplicáveis aos elementos de um conjunto universo $\U$, e $A$, $B$ e $C$ os subconjuntos de $\U$ dos elementos que satisfazem $P$, $Q$ e $R$, respectivamente. Expresse, em termos de implicações entre $P$, $Q$ e $R$, as seguintes relações entre os conjuntos $A$, $B$ e $C$.
  \begin{enumerate}[a)]
    \item $A \inter B^C \contido C$;
    \item $A^C \uniao B^C \contido C$;
    \item $A^C \uniao B \contido C^C$;
    \item $A^C \contido B^C \uniao C$;
    \item $A \contido B^C \uniao C^C$.
  \end{enumerate}
\end{exercise}

\begin{exercise}
  Considere as seguintes (aparentes) equivalências lógicas:
  \begin{align*}
    x = 1 & \sse x^2 -2x +1 = 0         \\
          & \sse x^2 -2 \vezes 1 +1 = 0 \\
          & \sse x^2 - 1 = 0            \\
          & \sse x = \maisoumenos 1
  \end{align*}
  Conclusão (?): $x = 1 \sse x = \maisoumenos 1$. Onde está o erro?
\end{exercise}

\begin{exercise}
  \label{exe:escrever-reciprocas}
  Escreva as recíprocas, contrapositivas e negações matemáticas das seguintes afirmações:
  \begin{enumerate}[a)]
    \item Todos os gatos têm rabo; $\left(G \implica R \right)$\\
    \sub{Recíproca:} Se têm rabo então é gato; $\left(R \implica G \right)$\\
    \sub{Contrapositiva:} Se não tem rabo então não é gato; $\left(\sim R \implica \sim G \right)$\\
    \sub{Negação:} Existe um gato que não tem rabo. $\left(G \land \sim R \right)$
    \item Sempre que chove, eu saio de guarda-chuva ou fico em casa;
    \item Todas as bolas de ping pong são redondas e brancas;
    \item Sempre que é terça-feira e o dia do mês é um número primo, eu vou ao cinema;
    \item Todas as camisas amarelas ou vermelhas têm manga comprida;
    \item Todas as coisas quadradas ou redondas são amarelas e
    vermelhas.
  \end{enumerate}
\end{exercise}

\begin{exercise}
  Considere os conjuntos: $F$ composto por todos os filósofos; $M$ por todos os matemáticos; $C$ por todos os cientistas; e $P$ por todos os professores. Exprima cada uma das afirmativas abaixo usando a linguagem de conjuntos:
  \begin{enumerate}[(i)]
    \item Todos os matemáticos são cientistas;
    \item Alguns matemáticos são professores;
    \item Alguns cientistas são filósofos; 
    \item Todos os filósofos são cientistas ou professores;
    \item Nem todo professor é cientista.
    \item Alguns matemáticos são filósofos;
    \item Nem todo filósofo é cientista;
    \item Alguns filósofos são professores;
    \item Se um filósofo não é matemático, ele é professor;
    \item Alguns filósofos são matemáticos.
  \end{enumerate}
  Tomando as cinco primeiras afirmativas como hipóteses, verifique quais das afirmativas restantes são necessariamente verdadeiras.
\end{exercise}

\begin{exercise}
  Em uma caixa fechada havia 5 bolas, Manoel abriu a caixa, vizualizou as bolas e declarou
  \begin{center}
    \aspas{Se tem uma bola preta então tem uma bola branca.}
  \end{center}

Em seguida, Márcio colocou a mão na caixa puxando uma bola branca, e Maurício, ao perceber, gritou
  \begin{center}
    \aspas{Então tem uma bola preta!}
  \end{center}

Supondo que Manoel está dizendo a verdade, o que podemos afirmar sobre a declaração de Maurício?
\end{exercise}

\begin{exercise}
  Considere um grupo de 4 cartões, que possuem uma letra escrita em um dos lados e um número do outro. Suponha que seja feita, sobre esses cartões, a seguinte afirmação: \emph{Todo cartão com uma vogal de um lado tem um número ímpar do outro}. Quais dos cartões abaixo você precisaria virar para verificar se essa afirmativa é verdadeira ou falsa?
  \begin{center}
    \begin{tabular}{|c|c|c|c|c|c|c|}
      \cline{1-1} \cline{3-3} \cline{5-5} \cline{7-7}
      A & $\empty$ & 1 & $\empty$ & B & $\empty$ & 4 \\
      \cline{1-1} \cline{3-3} \cline{5-5} \cline{7-7}
    \end{tabular}
  \end{center}
\end{exercise}

\begin{exercise}
  Numa mesa há cinco cartas:
  \begin{center}
    \begin{tabular}{|c|c|c|c|c|c|c|c|c|}
      \cline{1-1} \cline{3-3} \cline{5-5} \cline{7-7} \cline{9-9}
      Q & $\empty$ & T & $\empty$ & 3 & $\empty$ & 4 & $\empty$ & 6 \\
      \cline{1-1} \cline{3-3} \cline{5-5} \cline{7-7} \cline{9-9}
    \end{tabular}
  \end{center}
  Cada carta tem um número natural de um lado e uma letra de outro lado. Nico afirma: ``Qualquer carta que tenha uma vogal tem um número par do outro lado''. Jorel provou que Nico mente virando somente uma das cartas. Qual das cinco cartas Jorel teve que virar para provar que Nico mentiu?
\end{exercise}

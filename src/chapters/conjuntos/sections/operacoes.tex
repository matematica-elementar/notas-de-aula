\section{Operações}
\label{sec:op}
Assim como estudamos na aritmética operações entre números, como a adição e a multiplicação, vamos conhecer nesta seção algumas operações entre conjuntos. De uma forma geral, operações tem o objetivo de, a partir de um ou mais objetos de um determinado tipo, gerar um novo objeto que não necessariamente deve ser diferente do(s) original(is). Na multiplicação de números naturais, por exemplo, se operarmos os números dois e três, obtemos o número seis que é diferente de ambos. Mas se operarmos os números um e três, obtemos o número três, que é igual a um dos números originais.

Apresentaremos a seguir alguns exemplos de operações entre conjuntos, ou seja, que geram conjuntos novos a partir de conjuntos dados, bem como suas propriedades.

\import{}{uniao-e-intersecao.tex}
\import{}{complementar.tex}
\import{}{diferenca.tex}



\subsubsection{Gráfico}

\begin{example}
O gráfico da função quadrática $f(x) = ax^2$ é uma parábola cujo
foco é $F = \paren{0, \frac 1 {4a}}$ e cuja diretriz é a reta
horizontal $y = -\frac{1}{4a}$. Ademais, o vértice da parábola é a
origem do plano cartesiano.
\end{example}

\begin{proposition}
O gráfico de uma função quadrática $f(x) = ax^2 + bx + c$ é uma
parábola, tem a reta $x = -\frac {b}{2a}$ como eixo de simetria e o
ponto $\paren{-\frac {b} {2a}, -\frac {\Delta} {4a}}$ é o vértice da
parábola.
\end{proposition}

\begin{onlineact}
\khan{https://pt.khanacademy.org/math/algebra-home/alg-quadratics/quad-standard-form-alg/e/key-features-quadratics}
{Problemas com Expressões do Segundo Grau (Forma Padrão)}
\end{onlineact}

\begin{onlineact}
\khan{https://pt.khanacademy.org/math/algebra-home/alg-quadratics/alg-features-of-quadratic-functions/e/graphing_parabolas_2}
{Faça o Gráfico de Parábolas em Todas as Formas}.
\end{onlineact}

\begin{onlineact}
\khan{https://pt.khanacademy.org/math/algebra-home/alg-quadratics/alg-transforming-quadratic-functions/e/shift-parabolas}
{Deslocamento de Parábolas}.
\end{onlineact}

\begin{onlineact}
\khan{https://pt.khanacademy.org/math/algebra-home/alg-quadratics/alg-transforming-quadratic-functions/e/stretch-and-shrink-parabolas}
{Dimensionar e Refletir Parábolas}.
\end{onlineact}
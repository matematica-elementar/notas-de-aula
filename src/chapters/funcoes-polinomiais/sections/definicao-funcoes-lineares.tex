\subsubsection{Definição}

\begin{definition}
    Chamamos de \textdef{função linear} uma função real com lei de formação
$f(x) = ax$ para algum $a \in \R$. Quando $a = 0$, dizemos que $f$ é \textdef{identicamente nula}.
\end{definition}

\begin{remark}
Note que, sabendo que uma função é linear, o valor de $a$ é igual a
$f(1)$.
\end{remark}

No caso das grandezas inversamente proporcionais, a função
matemática que modela tal problema é uma função $f: \R^* \to
\R^*$ tal que $f(x) = \frac a x$. 
Nesse caso, também temos a
particularidade de que $f(1) = a \in \R^*$.

\begin{theorem}[Teorema Fundamental da Proporcionalidade]
\label{def:teorema-fundamental-proporcionalidade}
Seja $f: \R \to \R$ uma função crescente tal que $f(1)>0$. 
As seguintes afirmações são equivalentes:
%
\begin{enumerate}[(i)]
  \item $f$ é linear;
  \item $f(x+y) = f(x) + f(y)$ para quaisquer $x, y \in \R$;
  \item $f(nx) = nf(x)$ para todo $n \in \Z$ e todo $x \in \R$.
\end{enumerate}
\end{theorem}

\begin{remark}
    Um resultado análogo ao do Teorema \ref{def:teorema-fundamental-proporcionalidade} é válido
    no caso de $f$ ser crescente e tal que $f(1) < 0$.
\end{remark}

A importância do Teorema \ref{def:teorema-fundamental-proporcionalidade} 
está no fato de que, se quisermos saber se $f: \R \to \R$ é uma função linear não identicamente nula,
basta verificar duas coisas:
%
\begin{enumerate}[1.]
  \item $f$ é crescente ou decrescente;
  \item $f(nx) = n f(x)$ para todo $x \in \R$ e todo $n \in \Z$. No
  caso de o domínio e o contradomínio de $f$ serem o conjunto $\R^+$, basta verificar a condição
  para $n \in \N$.
\end{enumerate}

\begin{example}
O lado de um quadrado é proporcional à sua área? Em outras palavras,
essas duas grandezas podem ser relacionadas por meio de uma função
linear?
\end{example}
\subsubsection{Funções Afins e PAs}

\begin{proposition}
Seja  $f: \R \to \R$. Se $f$ é uma função afim e $\paren{x_1, x_2,
\dots , x_i, \dots}$ é uma PA, então a sequência formada pelos
pontos $y_i = f(x_i)$, $i \in \N^*$ é uma PA. Reciprocamente,
se $f$ for monótona e transformar qualquer PA $\paren{x_1, x_2,
\dots , x_i, \dots}$ numa PA com termo geral $y_i = f(x_i)$, $i \in
\N^*$, então $f$ é uma função afim.    
\end{proposition}

\begin{onlineact}
    \khan{https://pt.khanacademy.org/math/cc-eighth-grade-math/cc-8th-linear-equations-functions/8th-linear-functions-modeling/e/constructing-and-interpreting-linear-functions}
    {Problemas de Modelos de Funções Lineares}.
\end{onlineact}

\begin{onlineact}
    \khan{https://pt.khanacademy.org/math/cc-eighth-grade-math/cc-8th-linear-equations-functions/constructing-linear-models-real-world/e/constructing-linear-functions-word-problems}
    {Problemas de Como Escrever Funções}.
\end{onlineact}

\begin{onlineact}
    \khan{https://pt.khanacademy.org/math/cc-eighth-grade-math/cc-8th-linear-equations-functions/linear-nonlinear-functions-tut/e/linear-non-linear-functions}
    {Funções Lineares e Não Lineares}.
\end{onlineact}
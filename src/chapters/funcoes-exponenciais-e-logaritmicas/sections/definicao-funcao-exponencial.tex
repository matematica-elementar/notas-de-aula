\subsection{Definição}

\begin{definition}
    Seja $a$ um número real positivo diferente de 1. Chamamos de
\textdef{função exponencial} uma função $f: \R \to \R_+^*$ com lei de
formação $f(x) = a^x$. O número $a$ é chamado de \textdef{base} da função exponencial.
\end{definition}

\begin{definition}
Dizemos que uma função $f: \R \to \R$ é de \textdef{tipo exponencial}
quando $f(x) =b\cdot a^x$, onde $a,b \in\R$, $b$ é não nulo e $a$ é
positivo e diferente de 1.
\end{definition}

\begin{proposition}[Propriedades Fundamentais da Função Exponencial]
\label{prop:propriedades-funcao-exponencial}
Seja $f: \R \to \R_+^*$ uma função exponencial de base $a$.
Então, para quaisquer $x, y \in \R$ valem:
%
\begin{enumerate}[(i)]
  \item  $a^{x+y} = a^x\cdot a^y$, ou seja, $f(x+y) = f(x)\cdot f(y)$;
  \item $a^1 = a$, ou seja, $f(1) = a$;
  \item $x<y \implies \begin{cases} a^x < a^y, \ \ \ \text{ quando } \ \ \ a>1 \\
                                    a^y < a^x, \ \ \ \text{ quando } \ \ \ 0<a<1
                       \end{cases}.$
\end{enumerate}
\end{proposition}

Como consequências das propriedades listadas na Proposição \ref{prop:propriedades-funcao-exponencial},
pode-se listar os seguintes resultados acerca de uma função exponencial $f: \R \to \R_+^*$:
%
\begin{itemize}
    \item $f^{-1}(0) = \emptyset$, ou seja, $f$ não pode assumir o valor
    zero;
    \item $f(x)>0$, para todo $x \in \R$;
    \item Ao escolhermos o conjunto $\R_+^*$ como contradomínio de $f$, obtemos
    a sobrejetividade da função;
    \item $f$ é ilimitada superiormente;
    \item O gráfico de $f$ é uma linha contínua;
    \item $f$ é bijetiva e crescente se $a>1$, ou decrescente se
    $0<a<1$.
\end{itemize}

\begin{proof}
    Serão apresentadas as demonstrações dos dois primeiros itens, apenas. Os demais são consequências 
    imediatas dessas demonstrações, ou não estão do escopo deste texto.
    \begin{itemize}
        \item Suponha que $f(x_0) = 0$ para algum $x_0 \in \reais$.
        Agora, seja $y \in \reais$. Note que:
        %
        $$f(y) = f(x_0 + y - x_0) = f(x_0)\cdot f(y-x_0) = 0$$ 
        %
        Temos, em particular, que $f(1)=0$, porém $f(1) = a$, o que contradiz
        o fato de que a base da função exponencial não é nula. 
        Portanto, $f^{-1}(0) = \emptyset$.

        \item Seja $x \in \reais$. Note que:
        %
        $$f(x) = f\prn{\frac{x}{2} + \frac{x}{2}} = f\prn{\frac{x}{2}}\cdot f\prn{\frac{x}{2}} =\bracket{f\prn{\frac x 2}}^2$$
        %
        Como $f(x) \ne 0$ para todo $x \in \reais$, temos que $\bracket{f\prn{\frac x 2}}^2 > 0$. Portanto, $f(x) > 0$ 
        para todo $x \in \reais$.
    \end{itemize}
\end{proof}
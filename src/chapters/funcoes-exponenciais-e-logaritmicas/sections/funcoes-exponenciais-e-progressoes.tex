\subsection{Funções Exponenciais e Progressões}

\begin{proposition}
Seja  $f: \R \to \R$. Se $f$ é uma função do tipo exponencial e
$\paren{x_1, x_2, \dots , x_i, \dots}$ é uma PA, então a sequência
formada pelos pontos $y_i = f(x_i)$, $i \in \N^*$ é uma PG.
Reciprocamente, se $f$ for monótona injetiva e transformar qualquer
PA $\paren{x_1, x_2, \dots , x_i, \dots}$ numa PG com termo geral
$y_i = f(x_i)$, $i \in \N^*$ então $f$ é uma função real tal
que $f(x) = b \cdot a^x$ com $b = f(0)$ e $a = \frac {f(1)} {f(0)}$.
\end{proposition}

\begin{proof}
    Seja $f: \R \to \R$. Inicialmente, suponha que $f(x) = b a^x$, com $a,b\in\R$ tais que $a>0$ e $a\ne 1$.
    Além disso, suponha que $\seq{x_i}{i \in \nnats}$ é uma PA de razão $r$.
    
    Considere a sequência $\seq{y_i}{i \in \nnats}$ tal que $y_i = f(x_i)$ para cada $i \in \nnats$.
    Seja $n \in \nnats$. Temos que:
    %
    \[
        \frac{y_{n+1}}{y_n} = \frac{f(x_{n+1})}{f(x_n)} = \frac{\cancel b \cdot a^{x_{n+1}}}{\cancel b \cdot a^{x_n}} = 
        a^{x_{n+1}-x_n} = a^r
    \]
    %
    Logo, $\seq{y_i}{i \in \nnats}$ é uma PG.

    Reciprocamente, suponha que $f$ é monótona e injetiva e transforma qualquer PA $\seq{x_i}{i \in \nnats}$
    em uma PG $\seq{y_i}{i \in \nnats}$ tal que $y_i = f(x_i)$ para todo $i \in \nnats$.
    Ademais, tome $b = f(0)$.

    Seja $\func g {\reais}{\reais}$ tal que $g(x) = \frac{f(x)}b$. 
    Temos que $g$, assim como $f$, transforma PAs em PGs (exercício para o leitor).
    Agora, considere $a = g(1)$. 
    Note que $g(1) = \frac{f(1)}b$, o que implica que $a = \frac{f(1)}b$ e, consequentemente,
    $f(1) = b\cdot a$. 
    
    Note que, para qualquer $x \in \reais$, $-x$, $0$ e $x$ estão em PA. 
    Assim, $g(-x)$, $g(0)$ e $g(x)$ estão em PG. 
    Como $g(0) = \frac{f(0)}{b} = \frac{f(0)}{f(0)} = 1$, essa PG tem razão
    $\frac{g(x)}{g(0)} = g(x)$.
    Logo, $g(x) = \frac{1}{g(-x)}$, o que implica que:
    %
    \[
        g(-x) = \frac{1}{g(x)} = \prn{g(x)}^{-1}
    \]
    %
    Ademais, a PG $\prn{g(0)=1, g(x), g(2x), \dots, g(nx), \dots}$ tem razão $g(x)$.
    Logo, $g(nx) = \prn{g(x)}^n$ para todos $n \in \nats$ e $x \in \reais$.
    Como temos, também, que $g(-x) = \prn{g(x)}^{-1}$, podemos concluir que
    $g(nx) = \prn{g(x)}^n$ para todos $n \in \ints$ e $x \in \reais$.
    Segue, do Teorema~\ref{theo:caracterizacao-funcao-exponencial}, que $g(x) = a^x$, 
    em que $a = g(1)$. Assim,
    %
    \[
        f(x) = b\cdot g(x) = b \cdot a^x,
    \]
    %
    com $f(0) = b$ e $f(1) = b\cdot a$, ou seja, $a = \frac{f(1)}{f(0)}$.


\end{proof}
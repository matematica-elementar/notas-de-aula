\subsection{Gráfico}

\begin{example}
	Seja $f: \R \to \R_+^*$ uma função exponencial tal que $f(x) = a^x$. Na Imagem~\ref{img:graficos-exponencial}, são mostrados os gráficos nos casos de $a > 1$ e $0 < a < 1$.
	\begin{figure}[H]
		\centering
		\includegraphics[scale=0.30]{\imgdirfromsection/grafico-exponencial.png}
		\caption{Gráficos da função $f$ nos casos $0<a<1$ e $a>1$.}
		\label{img:graficos-exponencial}
	\end{figure}
	O gráfico de $f$ nunca toca o eixo $x$, mas fica tão próximo quanto queiramos. Isso equivale dizer que a reta $y=0$ é \emph{assíntota} do gráfico de $f$.
\end{example}

\begin{remark}
	Quando as bases de duas funções exponenciais são uma o inverso multiplicativo da outra, o gráfico de uma das funções consiste em uma reflexão do gráfico da outra. Em outras palavras, se $f:\reais \to \preais$ é uma função exponencial tal que $f(x) = a^x $, então a função $g:\reais \to \preais$ com regra $g(x) = \paren{\frac 1 a}^x$ é tal que:
	\[
		g(x) = \paren{\frac 1 a}^x = a^{-x} = f(-x),    
	\]
	e, conforme visto na Seção \ref{sec:dilatacao-e-reflexao}, o gráfico de $g$ é obtível a partir do gráfico de $f$ refletindo-o em relação ao eixo $y$. Ademais, note que, como $a > 0$ e $a \ne 1$, temos que:
	\[
		a > 1 \iff 0 < \frac{1}{a} < 1.
	\]
\end{remark}

\begin{example}
	O crescimento exponencial supera o de qualquer polinômio. Ao compararmos, por exemplo, as funções $f(x) = 2^x$ e $p(x)=x^{10}$, temos que:
	\begin{align*}
		0<x<1{,}077 & \implies  2^x > x^{10} \\
		1{,}078 < x < 58{,}77 & \implies  x^{10} > 2^x \\
		x>58{,}78 & \implies  2^x > x^{10}
	\end{align*}
	Na Imagem~\ref{img:polinomial-vs-exponencial}, são mostrados os gráficos das funções. Pode-se perceber que, a partir de certo valor de $x$, o valor da função exponencial fica maior que o da polinomial. 
	\begin{figure}[H]
		\centering
		\includegraphics[scale=0.30]{\imgdirfromsection/polinomial-vs-exponencial.png}
		\caption{Gráficos das funções $f$ e $p$.}
		\label{img:polinomial-vs-exponencial}
	\end{figure}
\end{example}

\begin{onlineact}
	\khan{https://pt.khanacademy.org/math/algebra/introduction-to-exponential-functions/exponential-expressions-alg1/e/exponential-expressions-word-problems-algebraic}{Problemas (Algébricos) de Expressões Exponenciais}.
\end{onlineact}

\begin{onlineact}
	\khan{https://pt.khanacademy.org/math/algebra/introduction-to-exponential-functions/exponential-decay-alg1/e/graphs-of-basic-exponential-functions}{Representação Gráfica de Crescimento e Decaimento Exponencial}.
\end{onlineact}

\begin{onlineact}
	\khan{https://pt.khanacademy.org/math/algebra2/exponential-and-logarithmic-functions/graphs-of-exponential-functions/e/graphs-of-exponential-functions}{Gráficos de Funções Exponenciais}.
\end{onlineact}
\section{Inequação do 2° grau}

\begin{definition}
    Uma \textdef{inequação do segundo grau} é uma relação de uma das formas
    a seguir:
    \begin{gather*}
        ax^2 +bx + c <0;\\
        ax^2 +bx + c>0;\\
        ax^2 +bx + c \le 0;\\
        ax^2 +bx + c \ge 0;
    \end{gather*}
    onde $a, b, c \in \R$, com $ a \ne 0$.
\end{definition}

\begin{example}
    Resolva as seguintes inequações:
    \begin{enumerate}[a)]
        \item $x^2 -3x +2 > 0$;
        \item $x^2 -3x +2 \le 0$.
    \end{enumerate}
\end{example}

\begin{solution}
    Observe que: 
    \[
        x^2 - 3x + 2 = (x-2)(x-1) > 0.
    \]
    Logo, teremos:
    \begin{figure}[H]
        \centering
        \includegraphics{\imgdirfromsection/[ok]photo_2018-08-24_22-55-35.jpg}
        \caption{}
    \end{figure}

    Assim, $S = \set{x \in \R \tq{x<1 \text{ ou } x>2}}$. Ademais, a solução $\bar{S}$ para a inequação $x^2-3x+2\le0$ será:
    \begin{align*}
        \bar{S} &= \set{x \in \R \tq x \ge 1 \text{ e } x \le 2} \\
                &= \set{x \in \R \tq 1 \le x \le 2} 
    \end{align*}
\end{solution}

\begin{example}
    Prove que a soma de um número positivo com seu inverso é sempre maior ou igual a 2.
\end{example}

\begin{proof}
    Queremos provar que $x+1/x\ge 2$, para todo $x > 0$. Note que:
    \begin{align*}
        x+\dfrac{1}{x}\ge 2 &\iff \dfrac{x^2+1-2x}{x}       \\
                            &\iff \dfrac{\prn{x-1}^2}{x}
    \end{align*}

    \begin{figure}[H]
        \centering
        \includegraphics{\imgdirfromsection/[ok]photo_2018-08-24_22-55-35(2).jpg}
        \caption{}
    \end{figure}

    Logo, para que $x+1/x\ge 2$, é necessário e suficiente que $x > 0$, ou seja, $x$ é positivo.
\end{proof}
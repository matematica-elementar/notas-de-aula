\section{Desigualdades Clássicas}

Para iniciar, apresentamos algumas desigualdades simples mas famosas, válidas para quaisquer $a,b \in \R$:
\begin{itemize}
  \item $\modu a \ge 0$;
  \item $a^2 \ge 0$;
  \item $\modu {a+b} \le \modu a + \modu b$ (desigualdade triangular).
\end{itemize}

\begin{theorem}
\label{theorem:ineq-prod-quad}
Para quaisquer $x, y \in \R$, vale:
%
\begin{equation*}
    xy \le \frac {x^2 +y^2} 2.
\end{equation*}
%
Além disso, a igualdade acontece se, e somente se, $x=y$.
\end{theorem}

\begin{proof}
Sejam $x, y \in \R$. Sabemos que $(x-y)^2 \ge 0$. Segue que:
%
\begin{align*}
	(x-y)^2 \ge 0 \iff & x^2 - 2xy + y^2 \ge 0 \\
				  \iff & 2xy \le x^2 + y^2 \\
				  \iff & xy \le \frac {x^2 + y^2} 2
\end{align*}
%
Ademais, note que a igualdade $xy = \frac {x^2 + y^2} 2$ ocorre quando:
%
\begin{align*}
	xy = \frac {x^2 + y^2} 2 \iff & (x-y)^2 = 0 \\
							 \iff & x-y = 0 \\
							 \iff & x = y
\end{align*}
\end{proof}

\begin{theorem}
\label{theo:desigualdade-medias-dois-termos}
Para quaisquer $a, b \in \R_+$, vale:
%
\begin{equation*}
    \sqrt{ab} \leq \frac {a +b} 2.
\end{equation*}
Além disso, a igualdade acontece se, e somente se, $a=b$.
\end{theorem}

\begin{proof}
Sejam $a, b \in R_+$. Para provar o teorema, basta aplicar o Teorema \ref{theorem:ineq-prod-quad} com $x = \sqrt a$ e $y = \sqrt b$.
\end{proof}

\begin{theorem}[Desigualdade das médias aritmética e geométrica]
\label{theorem:ineq-avg-ari-geo}
Para quaisquer $n \in \nnats$ e $a_1, a_2, \dots , a_n \in \R_+$, vale:
%
\begin{equation*}
    \sqrt[n]{a_1\dots a_n} \leq \frac {a_1 + \dots + a_n} n.
\end{equation*}
\end{theorem}

\begin{theorem}[Desigualdade das médias harmônica e geométrica]
Para quaisquer $n \in \nnats$ e $a_1, a_2, \dots , a_n \in \R_+^*$, vale:
%
\begin{equation*}
    \frac n {\frac 1 {a_1} + \dots + \frac 1 {a_n}}  \leq \sqrt[n]{a_1\dots a_n}  .
\end{equation*}
\end{theorem}

\begin{proof}
Sejam $a_1, a_2, \dots, a_n \in \R_+ ^*$. Considere $b_i = \frac 1 {a_i}$ para todo $i \in \set{1, 2, \dots, n}$. Usando o Teorema \ref{theorem:ineq-avg-ari-geo} com todos os $b_i$, tem-se que:
%
\begin{align*}
\sqrt[n]{b_1 \cdot \dots \cdot b_n } \le \frac {b_1 + \dots + b_n } n & \iff \frac n {b_1 + \dots + b_n } \le \frac 1 {\sqrt[n]{b_1 \cdot \dots \cdot b_n }} \\
& \iff \frac n {\frac 1 {a_1} + \dots + \frac n {a_n}} \le \frac 1 {\sqrt[n]{\frac 1 {a_1 \cdot \dots \cdot a_n}}} \\ 
& = \frac 1 {\frac 1 {\sqrt[n]{a_1 \cdot \dots \cdot a_n}}}\\ 
& =  \sqrt[n]{a_1 \cdot \dots \cdot a_n}
\end{align*}
\end{proof}

\begin{theorem}[Desigualdade de Cauchy-Schwarz]
Sejam $x_1, \dots , x_n, y_1, \dots y_n \in \R$. O seguinte vale:
%
\begin{equation*}
    \modu{x_1y_1 + \dots + x_ny_n} \leq \sqrt{x^2_1+ \dots + x^2_n}
    \cdot \sqrt{y^2_1+ \dots + y^2_n}.
\end{equation*}
%
Além disso, a igualdade só ocorre se existir um número real $\alpha$ tal que $x_1 = \alpha y_1$, ..., $x_n = \alpha y_n$.
\end{theorem}

\begin{example}
Duas torres são amarradas por uma corda $APB$ que vai do topo $A$ da primeira torre para um ponto $P$ no chão, entre as torres, e então
até o topo $B$ da segunda torre. Qual a posição do ponto $P$ que nos dá o comprimento mínimo da corda a ser utilizada?
\end{example}

\begin{solution}
Tomando $B'$ como o reflexo de $B$ em relação ao chão, conforme a Figura \ref{fig:torres}, temos que o comprimento da corda $\overline{AP} + \overline{PB}$ é igual a $\overline{AP} + \overline{PB'}$. 

\label{fig:torres}
\begin{figure}[H]
\includegraphics[scale=0.25]{\imgdirfromsection/fig-c03-ex29.png}
\centering
\end{figure}

Traçando $AB'$ e tomando $P'$ como a interseção de $AB'$ com o chão, tem-se que $P'$ é a solução do problema pois, pela desigualdade triangular, segue que:
%
\begin{align*}
\overline{AP'} +\overline{P'B} & = \overline{AP'} +\overline{P'B'} \\
& = \overline{AB'} \\
& \le \overline{AP} +\overline{PB'} \\
& = \overline{AP} +\overline{PB} 
\end{align*}
%
Logo, $\overline{AP'} +\overline{P'B} \le \overline{AP} +\overline{PB}$.
\end{solution}

\begin{example}
Prove que, num triângulo retângulo, a altura relativa à hipotenusa é sempre menor ou igual que a metade da hipotenusa. Prove, ainda, que
a igualdade só ocorre quando o triângulo retângulo é isósceles.
\end{example}

\begin{solution}
Considere um triângulo retângulo como o da Figura \ref{fig:rect-triangle}.
%
\label{fig:rect-triangle}
\begin{figure}[H]
\includegraphics{\imgdirfromsection/03-triangle.png}
\centering
\end{figure}
%
Queremos provar que $h \le \frac a 2$. Da semelhança entre $AHC$ e $BAC$, segue que:
%
\begin{equation*} 
\frac b h = \frac a c \iff ah = bc
\end{equation*} 
%
Do Teorema \ref{theorem:ineq-prod-quad}, segue que:
%
\begin{align*}
ah = bc \le \frac {b^2 + c^2} 2 = \frac {a^2} 2,
\end{align*}
%
ou seja, 
%
\begin{align*}
ah \le \frac {a^2} 2 \iff h \le \frac a 2.
\end{align*}
%
Além disso, a igualdade só será válida quando $b=c$, ou seja, quando o triângulo for isósceles.
\end{solution}

\begin{example}
Prove que, entre todos os triângulos retângulos de catetos $a$ e $b$, e com hipotenusa $c$ fixada, o que tem maior soma dos catetos
$S = a+b$ é o triângulo isósceles.
\end{example}

\begin{solution}
Seja um triângulo retângulo com hipotenusa $c$ fixada e catetos $a$ e $b$. Utilizando a Desigualdade de Cauchy-Schwarz com $x_1 = a$, $x_2 = b$, $y_1 = 1$ e $y_2 = 1$, temos que:
%
\begin{align*}
\modu {a \cdot 1 + b \cdot 1} \le \sqrt{a^2 + b^2} \cdot \sqrt{1^2 + 1^2} \iff S = a+b \le c\sqrt 2.
\end{align*}
%
Além disso, da Desigualdade, segue que $S$ é igual a $c\sqrt 2$, ou seja, atinge seu valor máximo se, e somente se, $a = \alpha \cdot 1$ e $b = \alpha \cdot 1$ para certo $\alpha \in \R$. Logo, o triângulo deve ser isósceles, com $a=b$.
\end{solution}
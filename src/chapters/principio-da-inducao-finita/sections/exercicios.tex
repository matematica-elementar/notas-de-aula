\section{Exercícios}

\begin{exercise}
Demonstre, por indução, que para qualquer $n \in \N^*$ é válida a igualdade:
$$1+2+ 3+ \dots + n = \frac{n \paren {n+1}} 2.$$
\end{exercise}

\begin{exercise}
Demonstre, por indução, que para qualquer $n \in \N^*$ são válidas as igualdades:
\begin{enumerate}[a)]
	\item $$1^2 +2^2 + 3^2+ \dots + n^2 = \frac{n \paren {n+1} \paren{2n+1}} 6.$$
	\item $$2^2 + 4^2 + 6^2 + \dots + (2n)^2 = \dfrac{2n(2n+1)(n+1)}{3};$$
	\item $$1^2 + 3^2 + 5^2 + \dots + (2n-1)^2 = \frac{n(2n-1)(2n+1)}{3}.$$
\end{enumerate}
\end{exercise}

\begin{exercise}
	Mostre, por indução, para todo $n \in \mathbb N^\ast $, que 
	$$(n+1)(n+2)(n+3) \dots (n+n) = 2^n \cdot 1 \cdot 3 \cdot 5 \dots (2n-1) .$$
\end{exercise}

\begin{exercise}
	Mostre, por indução, para todo $n \in \mathbb N^\ast $, que
	$$1!\cdot 1 + 2! \cdot 2 + 3! \cdot 3 + \dots + n! \cdot n = (n+1)! - 1.$$
\end{exercise}

\begin{exercise}
	Mostre, usando o Princípio da Indução Finita, que ao somarmos três números naturais consecutivos elevados ao cubo, obtemos um múltiplo de 9. Ou seja,
	$$n^3 + (n+1)^3 + (n+2)^3 = 9c,$$
para todo $n \in \N $ e algum $c \in \N$.
\end{exercise}

\begin{exercise}
Prove que $3^{n-1} < 2^{n^2}$ para todo $n \in \N^*$.
\end{exercise}

\begin{exercise}
Mostre, por indução, que
$$\paren{\frac {n+1}{n}}^n \leq n,$$
para todo $n \in \N^*$ tal que $n \geq 3$.

\begin{hint}
Mostre que $\frac{k+2}{k+1} \leq \frac{k+1} k$ para todo $k \in
\N^*$. Depois, eleve tudo à potência $k+1$.
\end{hint}
\end{exercise}

\begin{exercise}
Prove que
$$\frac 1 {\sqrt 1} +\frac 1 {\sqrt 2} +\frac 1 {\sqrt 3} + \dots + \frac 1 {\sqrt n} \geq \sqrt n,$$
para todo $ n \in \N^*$.
\end{exercise}

\begin{exercise}\label{exercicio:polinomio}
	Prove, usando o Princípio da Indução Finita, que $n^2 - 7n +12 \geq 0$ para todo $n \in \N$ tal que $n \geq 3$.
\end{exercise}

\begin{exercise}
	Mostre, por indução, que vale a seguinte inequação para todo $n \in \N$, tal que $n > 1$,
	$$\dfrac 1 {n+1} + \dfrac 1 {n+2} + \cdots + \dfrac 1 {2n} > \dfrac {13}{24}.$$
\end{exercise}

\begin{exercise}
	Sejam $a \in \R$ e $b \in \R$ tais que $a+b>0$ e $a \neq b$. Mostre, para todo $n \in \N^\ast$, que:
            $$ 2^{n-1}(a^n + b^n) \geq (a+b)^n .$$
\end{exercise}

\begin{exercise}
	Sejam $n \in \mathbb{N}$ tal que $n >1$ e $A_i$ conjuntos para todo $i \in \mathbb{N}^\ast$. Mostre, por indução finita, que $$\left( A_1 \cup A_2 \cup \dots \cup A_n \right)^C = A_1^C \cap A_2 ^C \cap \dots \cap A_n^C.$$
\end{exercise}

\begin{exercise}
Um subconjunto do plano é \emph{convexo} se o segmento ligando quaisquer dois de seus pontos está totalmente nele contido.
Os exemplos mais simples de conjuntos convexos são o próprio plano e qualquer semi-plano.
Mostre que, para qualquer $n \in \N^*$, a interseção de $n$ conjuntos convexos é ainda um conjunto convexo.
\end{exercise}

\begin{exercise}
Diz-se que três ou mais pontos são \emph{colineares} quando eles todos pertencem a uma mesma reta.
Do contrário, diz-se que eles são \emph{não colineares}.
Além disso, dois pontos determinam uma única reta.
Usando o Princípio da Indução Finita mostre que $n$ pontos, $n\geq 3$, tais que quaisquer 3 deles são não colineares, determinam
$$\frac{n!}{2\cdot(n-2)!}$$
retas distintas.
\end{exercise}
